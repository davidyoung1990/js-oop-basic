\chapter{JavaScript\,与面向对象}

本章要点

\begin{itemize}
\item 解释什么是对象, 什么是面向对象, 以及面向对象编程是什么.
\item 能解释什么是面向过程编程, 什么是面向对象编程.
\item 能利用面向对象的方式对需求进行分析, 抽取出需要的对象.
\item 能简要分析面向对象与面向过程之间的关系.
\end{itemize}


%%%%%%%%%%%%%%%%%%%%%%%%%%%%%%%%%%%%%%%%%%%%%%%%%%%%%%%%%%%%%%%%%%%%%%%%%%%%%%%%%%%%%%%%

\section{什么是对象}

所谓对象(\,object\,), 即目标. 就是你所研究的目标, 具体的东西. 在前面我们已经学习过一些\,\emph{内置对象}\,了. 
例如: 数组, 字符串包装对象, 时间对象等. 你么我们回忆一下这些对象的使用. 同时思考一下对象的特征是什么?

%%%%%%%%%%%%%%%%%%%%%%%%%%%%%%%%%%%%%%%%%%%%%%%%%%%%%%%%%%%%%%%%%%%%%%%%%%%%%%%%%%%%%%%%

\subsection{内置对象的使用复习}

我们分别讨论一下数组, 字符串包装对象和时间对象.



\subsubsection{数组对象}

首先考虑数组%
%
\begin{lstlisting}
var arr = [ 1, 2, 3, 4 ];
\end{lstlisting}%
%
要对数组进行操作, 我们就需要使用数组提供的\,\emph{方法}, 例如要反转数组, 需要使用\,\jcodeinline{reverse}\,方法;
要提取数组中的某一个子字符串, 则需要使用\,\jcodeinline{slice}\,方法; 要判断数组中是否含有某一个字符, 则需要使用%
\,\jcodeinline{indexOf}\,方法%
%
\footnote{\jcodeinline{indexOf}\,方法是\,ES5\,中引入的方法, 对于低版本的\,IE\,浏览器是不支持的. 需要自行定义};%
%
如果需要往数组中添加一个元素, 则使用\,\jcodeinline{push}\,方法, 或\,\jcodeinline{unshift}\,方法; 如果是要从数组中移除某个元素,
可以使用\,\jcodeinline{pop}\,方法, \,\jcodeinline{shift}\,方法, 或\,\jcodeinline{splice}\,方法%
%
\footnote{其具体使用语法可以参考代码: \textsf{01-01-01-array.html}}.



\subsubsection{字符串包装对象}

然后我们讨论字符串的包装对象. 原本字符串是基本类型, 并不是对象. 但是在\,JavaScript\,中为了统一处理数据, 提供了基本类型的包装对象:
每一个基本类型依旧可以点出方法, 并调用得到结果(对于数字类型与布尔类型相对讨论较少, 但是对于字符串, 系统提供了很多的方法). 其本质是,
在调用的时候, 基本类型会转换成对象类型, 然后由对象调用对应的方法, 得到结果后对象就会被销毁. 这个过程虽然我们无法直接观察到, 
但是我们需要了解其过程.

对于字符串%
%
\begin{lstlisting}
var str = 'hello JavaScript string';
\end{lstlisting}%
%
如果要计算该字符串中有多少个单词, 可以使用\,\jcodeinline{split}\,方法, 用空格分隔字符串, 然后计算得到的数组结果的长度; 
如果要判断这个字符串中是否含有某一个子字符串, 则可以使用\,\jcodeinline{indexOf}\,方法; 
如果需要将字符串全部转换成大写或小写, 则可以使用\,\jcodeinline{toUpperCase}\,或\,\jcodeinline{toLowerCase}\,方法%
%
\footnote{其具体使用语法可以参考代码: \textsf{01-01-02-string.html}}.
很显然方法全部集中在一起, 然后需要使用变量点出来使用.




\subsubsection{时间对象}

前面我们讨论了数组, 字符串. 它们都是对象, 可以保存数据, 同时还有很多方法可以对数据进行操作. 接下来我们再来看看时间对象.

对于时间对象%
%
\begin{lstlisting}
var now = new Date();
\end{lstlisting}%
%
使用\,\jcodeinline{new Date()}\,创建一个时间对象, 
对象中存储了从\,1970\,年 1\,月 1\,日 0\,点 0\,分 0\,秒到现在的毫秒数,
即时间戳. 对象除了存储该时间戳外, 还提供了很多的方法. 
例如: 要获得当前时间的年份数, 可以使用\,\jcodeinline{getFullYear}\,方法;
如果要获得当前的月份\,\jcodeinline{getMonth}\,方法%
%
\footnote{月份的计算从\,0\,开始, 即返回\,0\,表示一月份, 返回\,11\,表示第十二月份}.%
%
如果要获得当前时间的小时, 分钟以及秒数, 可以使用\,\jcodeinline{getHours}\,方法, \jcodeinline{getMinutes}\,方法,
以及\,\jcodeinline{getSeconds}\,方法%
%
\footnote{其具体使用语法可以参考代码: \textsf{01-01-03-date.html}}.






\subsubsection{总结特征}

从面前的几个案例我们可以归纳出对象的特征:

\begin{enumerate}
\item 对象不是一个简单的数据类型, 其包含的东西很多(很直观的感觉).
\item 对象保存了数据, 数组对象保存的就是数组中的各个元素, 字符串的包装对象里面保存着字符串的数据, 时间对象保存的当前时间戳. 
        实际上对象可以保存很多数据, 不仅仅就这么一点.
\item 对象中保存着很多的方法, 而这些方法都是用来操作保存的数据用的. 例如, 数组的方法, 用于操作当前数组中的每一个元素; 
        字符串方法就是在操作当前字符串的每一个字符; 而时间方法实际上就是在时间戳上进行计算, 来转换成这个时间对应的年, 
        月, 日, 时, 分, 秒等.
\end{enumerate}

通过前面的小结. 我们不难想象: {\em 对象就是一个复合结构, 其中包含了数据和方法. 数据用于描述对象的特征与信息; 
而方法用途描述对象具有的能力与行为}. 



%%%%%%%%%%%%%%%%%%%%%%%%%%%%%%%%%%%%%%%%%%%%%%%%%%%%%%%%%%%%%%%%%%%%%%%%%%%%%%%%%%%%%%%%



\subsection{对象的概念}

对象(\,Object\,), 是一种数据的组织方式, 是一个抽象的概念. 是程序员组成代码的一种方法. 
前面我们复习了常见的内置对象, \textbf{\emph{对象就是数据与方法的综合体}}. 在早期的编程方法中,
数据单独定义, 函数也是单独定义, 将数据与操作数据的函数分开编写, 使用的时候在合到一起. 
这样对于小规模的程序问题并不算太糟. 但是当多人协作开发, 以及大量代码与功能组合后, 维护代码将变得十分困难.
例如, 当多人分别开发不同功能的时候, 变量的命名就有可能造成代码工作时出现问题. 
而针对\,JavaScript\,这一解释型编程语言, 重复定义变量又不会提示与警告, 这对调错又进一步增加了困难.

而将数据与功能``打包''到一起后, 就可以自己来维护与管理自己, 从而使得编码与维护变得相对容易. 
当然对于小规模的代码, 依旧会变得糟糕. 因为可能一句话就可以解决的事情, 由于对象的加入, 
可能需要好几句话才可以调用完成. 这也是初学者最容易困惑的地方. 但是也不用担心, 在积累一定代码与功能后, 
初学者也可以很快感受到使用对象来组织代码的便利之处. 

在接下来的篇幅中, 我们一一讨论面向对象的所有主题, 来体会面向对象所带来的优势.



%%%%%%%%%%%%%%%%%%%%%%%%%%%%%%%%%%%%%%%%%%%%%%%%%%%%%%%%%%%%%%%%%%%%%%%%%%%%%%%%%%%%%%%%


\section{面向对象编程}

在讨论面向对象编程之前, 我们需要了解一下什么是面向过程. 我们再对比面向过程与面向对象之间的的关系.

%%%%%%%%%%%%%%%%%%%%%%%%%%%%%%%%%%%%%%%%%%%%%%%%%%%%%%%%%%%%%%%%%%%%%%%%%%%%%%%%%%%%%%%%

\subsection{面向过程编程}

面向过程(\,Procedure Oriented\,), 就是以过程为中心的一种编程思想. 面向, 就是指使用, 关注, 以什么的方式的含义. 
面向过程就是以过程的方式, 通过处理每一件事情来处理代码. 那么过程就需要关注每一个步骤与细节, 对于开发者,
必须关注到每一件事情, 每一个函数的调用. 

用一个比较简单的例子来说明:
%
\begin{quote}
例如我们考虑一下吃早饭. 如果是在大年初一的早上. 我们要吃面怎么处理呢? 既然大年初一, 街上肯定没有小吃, 
只能自己做了. 所以我们需要考虑一下做面的步骤了.
%
\begin{enumerate}
\item 首先需要先和面. 把面饭和水混合到一起, 然后 揉啊揉, 揉啊揉, ...
\item 不知过去了多少时间, 然后将和好的面压成细面(这还需要压面机, 否则只有自己拉了 ... )
\item 然后生火烧水, 与此同时可以做点臊子(就是加载面中调味使用的配菜)
\item 待水烧开即可下面, 接着就是等待, 等待. 还需不时的捞一捞, 以免面粘锅.
\item 最后就可以迟到好吃的面条了. 只可惜最后还需要洗锅, 洗完 ...
\end{enumerate}
%
看来吃个面也不是那么容易的.
\end{quote}
%
说到这里, 不难发现这个与我们现在吃面的方式不太相同. 现在大多数人想要吃面, 直接找个面馆, 然后点一碗面即可.

在这个案例中, 就是面向过程的. 在这个制作面的过程中, 我们需要亲力亲为, 需要自己一步一步的处理, 不能先下面, 再和面;
也不能先吃面条再煮面. 所有的一切都说明每一个细节步骤都不能跳过, 也不能随便处理. 只有每一个步骤做到极致, 面才可口好吃.

结合到编码中, 就是在完成一段逻辑的时候, 考虑先做什么, 后做什么, 然后接着做什么. 即为面向过程的思考方式.

我们可以思考下面的案例:

\begin{jdemo}
    在页面中创建一个\,select\,标签, 标签中的选项为: 北京, 上海, 广州, 和深圳. 其取值分别为 1, 2, 3, 4.
\end{jdemo}

在这个案例中我们要创建一个\,select\,标签. 按照我们常规的想法:%
%
\begin{enumerate}
\item 首先调用\,\jcodeinline{document.createElement}\,方法创建\,select\,对象.
\item 然后接着调用\,\jcodeinline{document.createElement}\,方法创建\,option\,对象.
\item 接着使用对象的\,\jcodeinline{innerHTML}\,属性, 赋值内部文本.
\item 使用\,\jcodeinline{value}\,属性来设置选中后的取值.
\item 接着使用\,select\,对象的\,\jcodeinline{appendChild}\,方法, 来追加\,option\,对象.
\item 使用循环, 根据``北京'', ``上海'', ``广州'', ``深圳''数据循环生成\,option\,对象依次加入.
\item 最后将\,select\,对象加到页面中.
\end{enumerate}

\noindent
参考代码为\footnote{完整代码请参考: \textsf{01-02-01-select.html}}:%
%
\begin{lstlisting}
var select = document.createElement( 'select' );
var data = [ '`北京`', '`上海`', '`广州`', '`深圳`' ];
for ( var i = 0; i < data.length; i++ ) {
    var option = document.createElement( 'option' );
    option.value = i + 1;
    option.innerHTML = data[ i ];
    select.appendChild( option );
}
document.getElementById( 'dv' ).appendChild( select );
\end{lstlisting}%
%
运行结果为:\\
%
\jimg{imgs/2018-01-15_160606.png}

在这段代码中所有的考虑都是一步步完成的, 不能颠倒也不能省略某一个步骤. 这便是面向过程的编码方式.



%%%%%%%%%%%%%%%%%%%%%%%%%%%%%%%%%%%%%%%%%%%%%%%%%%%%%%%%%%%%%%%%%%%%%%%%%%%%%%%%%%%%%%%%


\subsection{面向对象编程}

面向对象编程(\,OOP, Object Oriented Programming\,), 就是使用对象进行编程. 即在需要某一个任务的时候, 
不再考虑自己亲力亲为, 而是找到一个对象, 要求对象帮助你完成. 

还是来考虑吃面的例子. 在面向对象的思想中, 要吃面了, 不是考虑怎么去做, 而是考虑先找到一个可以做面的对象,
即面馆. 然后告知你要吃什么面. 然后等待即可得到一碗符合你要求的面. 这就是面向对象的思想.

实际上在生活中我们时时刻刻的来利用对象帮我们完成所需:
%
\begin{itemize}
\item 我们要到某一个地方, 我们不需要了解先到哪里, 然后再怎么走. 叫一个出租车(\,滴滴\,)即可(\,面向滴滴出行\,).
\item 我们要买食材, 不需要到菜地里挑选, 找到菜市场全有了(\,面向菜市场购物\,).
%\item 我们要购物, 打开\,app\,挑选即可, 不用跑到商厦里面挑来挑去(\,面向程序购物, 当然也有人喜欢购物的感觉\,).
\item 我们要洗衣服的时候, 不用手洗, 找到洗衣机, 放入衣服和洗衣液, 按下按钮即可(\,面向洗衣机\,).
\item 我们要写代码, 不需要一步一步思考处理步骤, 找到对象, 告诉它要做什么, 它就会处理好. 
\item ... ...
\end{itemize}
%
这样的案例还有很多, 我们再来仔细看看使用\,OOP\,的方式如何创建\,select\,标签.

这里我们暂时只是来说明面向对象与面向过程的区别, 暂时不考虑技术, 所以代码只给出了使用对象的片段%
%
\footnote{完整的代码可以参考: \textsf{01-02-02-oopselect.html}}:
%
\begin{lstlisting}
var select = new Select({
    container: document.getElementById( 'dv' )
    , data: [ '`北京`', '`上海`', '`广州`', '`深圳`' ] 
});
select.init();
\end{lstlisting}%
%
其运行结果为:\\
%
\jimg{imgs/2018-01-15_165823.png}

\noindent
在这段代码中, 我们需要创建一个下拉列表, 即\,select\,标签. 我们只需要先找到\,\jcodeinline{Select}\,构造函数.
然后创建\,\jcodeinline{Select}\,对象. 同时告诉对象我们需要将创建的标签加到哪一个容器中. 
以及我应该使用什么数据来显示下拉列表.

这便是使用面向对象的思想在组织代码.


%%%%%%%%%%%%%%%%%%%%%%%%%%%%%%%%%%%%%%%%%%%%%%%%%%%%%%%%%%%%%%%%%%%%%%%%%%%%%%%%%%%%%%%%

\subsection{怎样抽取对象}

我们可以发现, 面向对象与面向过程有一个重要的不同点. 就是面向过程, 重点在考虑如何做? 每一步该怎么写?
而面向对象在考虑, 做什么? 什么对象可以以做? 

那么问题也来了, 在进行开发的时候怎么找对象呢? 

显然, 就需要学习和积累了. 如\,C\#, Java\, 等这一类编程语言(或平台), 提供了大量的类库. 
我们在学习和工作中不断积累. 了解各种库的使用方法与优劣. 在遇到需要解决的问题时自然就知道该用什么对象了.
但是如果按照当下我们的经验, 不足以知道应该使用什么库呢? 那就需要自己来进行编写对象了.

前面我们了解到, 所谓的对象就是数据与方法的结合体. 所以编写对象就是在编写需要存储什么数据, 
以及需要提供什么方法. 那么首先我们得知道我们需要什么对象. 看\,\figurename\ref{ch01_0001}, 
%
\begin{figure}[htb!p]
% \centering
\jimg{./imgs/2018-01-16_100451.png}

\caption{跳一跳小游戏\label{ch01_0001}}
\end{figure}
%
其中我们可以用自己的话来描述一下界面上都有什么东西. 
例如, 从上往下, 首先有一个大的名字: 叫``跳一跳'', 然后是一个方盒子(搁棋子用), 
接着是一个搁着棋子的方盒子, 然后一个按钮, 上面写着``开始游戏'', 最后还有一个排行榜.

用自己的话描述完这个界面后, 我们可以简单分析一下, 在我们描述的这段话中, 有几个名词?
%
\begin{enumerate}
\item 一个标题: ``跳一跳''
\item 两个方盒子
\item 一个棋子 
\item 开始游戏的按钮
\item 排行榜按钮
\end{enumerate}
%
一共\,6\,个名词. 那么我们可以得到\,6\,个对象. 

也就是说, 用自己的话描述界面, 有几个名词就可以得到几个对象. 我们称这个方法为\emph{名词提炼法}.

对象有了, 接下来就需要讨论一下对象有什么数据与方法了.

\begin{description}
\item[标题对象]
    标题对象中一个非常重要的数据就是标题内容. 即存储显示字符串. 除此之外还有坐标.
    该标题在页面的上部的正中间. 因此还需要存储坐标, 以及标题的宽度高度. 再细化,
    还需要存储字体, 前景颜色, 背景颜色等.

    因此该对象可以抽取出的数据有: 显示文本, 距离屏幕左边框的长度, 
    距离屏幕顶边框的长度, 字体, 前台颜色, 和背景颜色.
\item[盒子对象]
    屏幕中有两个盒子, 即两个盒子对象. 每一个盒子对象除了表面颜色, 大小不一样外, 
    其结构是相同的, 都是一个立方体. 
    
    因此可以抽取的数据有: 坐标, 矩形边长, 高, 以及外部贴图.
\item[棋子对象]
    棋子可以在界面中跳来跳去, 因此, 棋子有坐标. 除此之外棋子的外形也可以抽取出数据来.
    例如颜色, 外形曲线等\footnote{这里简化了棋子, 将其看成一个点}. 

    棋子可以跳, 因此它还具有\,\jcodeinline{jump}\,方法.
\item[按钮对象]
    界面中有两个按钮, 一个是开始游戏, 一个是排行榜. 都是点击后可以做指定的事情.

    开始游戏按钮点击后即可进入游戏. 从界面上看, 按钮上有一个绿色的图标, 有一段文本.
    因此开始按钮可以抽取出的数据有, 坐标, 文本, 字体, 前后背景色, 宽高, 还有图标.

    排行榜按钮与开始按钮类似, 也有图标和文字, 而且右边还有一个箭头. 
    同样可以抽取的数据有, 坐标, 文本, 字体, 前后背景色, 宽高, 还有图标.

    因此按钮可以抽取的数据有, 显示的文本, 显示的图标, 前景颜色, 背景颜色, 按钮宽度,
    按钮高度, 按钮距离屏幕左边与顶边的距离等.
\end{description}



我们可以看一个更为复杂的例子:

\begin{jdemo}
看下面植物大战僵尸的游戏截图:\\[0.5em]

\noindent\jimg{imgs/zhiwudazhanjiangshi.jpg}
\\[1em]
\noindent 在这个界面上有什么对象呢?

\end{jdemo}

简单分析这张游戏截图:
{\renewcommand{\labelitemi}{-}
\begin{itemize}
\item 整个界面背景可以看成一张大的铺满屏幕的图片, 可以抽取成一个对象. 当然由于铺满整个界面,
      因此也就不需要考虑其他数据了, 就一个图片数据.
\item 图片左上角有一个植物等待进度面板. 可以抽取成对象. 其包含的数据可以有: 坐标, 宽高,
      植物列表, 以及积累的太阳数量等. 
\item 下方有种植的植物---太阳花. 作为对象需要保存的数据有, 坐标, 宽高, 以及产生小太阳的时间间隔. 
      还有一个特殊的数据, 就是在僵尸吃植物的时候, 需要吃几次才可以消除该植物. 即还需要存储血量.
      同时太阳花提供生产小太阳的方法.
\item 除了太阳花, 还有豌豆, 豌豆的存储的数据与太阳花类似, 有坐标, 宽度与高度, 血量.
      存储的时间间隔不是用来生成小太阳花, 而是用来发射豌豆. 即发射豌豆的时间间隔.
      同时它还需要提供发射豌豆方法.
\item 地雷. 地雷需要的数据有坐标, 宽高, 以及等待的时间间隔. 同时提供爆炸方法.
\item 僵尸. 僵尸存储的有坐标, 移动速度等. 由于僵尸可以被豌豆等消灭, 所以僵尸也有血量. 
      僵尸所提供的的方法有移动和吃.
\end{itemize}
}
由于现在主要考虑的是如何抽取对象. 所以这里还可以继续分析其成员与方法. 由于本节主要是介绍怎么提取对象,
至于怎么实现对象还需要后续的课程学习.



%%%%%%%%%%%%%%%%%%%%%%%%%%%%%%%%%%%%%%%%%%%%%%%%%%%%%%%%%%%%%%%%%%%%%%%%%%%%%%%%%%%%%%%%

\section{面向对象与面向过程的关系}

前面我们已经看到, 面向对象编程总结起来就是, \emph{要做什么事情, 找到合适的对象, 告诉它去完成即可}.
很显然面向对象非常好, 那是不是就不需要面向过程了呢? 既然面向对象这么好, 那谁还去学习面向过程呢?
那么, 我们首先来看看使用面向对象的策略: 

\begin{enumerate}
\item 首先分析问题, 看系统是否提供相应对象可供使用. 如果有直接使用即可. 
\item 如果没有, 系统没有提供对应的功能, 查看是否有第三方提供相应的解决方案. 例如是否有相应的库, 
      可以帮助我们快速解决问题. 如果有, 下载下来直接使用. 
\item 若系统不支持, 第三方库也没有, 货支持的不够好, 那么我们就需要自己开发了.      
\end{enumerate}

很显然面在有内置对象或第三方对象时, 我们可以很轻松的面向对象开发. 什么意思呢?

\begin{itemize}
\item 例如, 要对数组排序, 我们不需要关系数组的排序算法是怎样的. 虽然我们已经熟练使用冒泡排序, 
      但是, 针对大量数据, 冒泡排序的性能还是非常非常低的. 因此在系统内部, 排序采用了一些策略.
      但是我们根本不用知道它怎么处理, 因为它已经面向对象了. 我们只要告诉它我们要排序即可.
\item 再比如, 我们要获得今天的年份数, 月份数, 或日期. 我们知道, 
      \jcodeinline{Date}\,对象存储了当前时间戳. 它是一个以毫秒为单位的数字. 如果要计算出年份,
      想一想该怎么算? 除以毫秒, 分钟, 小时, 天, ... 还要加上闰年? 很明显不是那么好计算. 
      但是我们不担心, 因为有方法, \jcodeinline{getFullYear}\,方法, \jcodeinline{getMonth}\,方法,
      以及\,\jcodeinline{getDate}\,等方法, 可以很容易的取得索要的数据. 根本不用担心如何计算
\item ... ...
\end{itemize}





%%%%%%%%%%%%%%%%%%%%%%%%%%%%%%%%%%%%%%%%%%%%%%%%%%%%%%%%%%%%%%%%%%%%%%%%%%%%%%%%%%%%%%%%

\section{小结}







%%%%%%%%%%%%%%%%%%%%%%%%%%%%%%%%%%%%%%%%%%%%%%%%%%%%%%%%%%%%%%%%%%%%%%%%%%%%%%%%%%%%%%%%

\begin{exercise}
\item 测试
\item 测试
\item 测试
\item 测试
\end{exercise}








